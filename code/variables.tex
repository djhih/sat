\documentclass[12pt,a4paper]{article}

% 版面設置
\usepackage[margin=1in]{geometry}
\usepackage{fancyhdr, lastpage}

% 數學與符號
\usepackage{amsmath, amssymb, amsthm, mathtools, braket, cancel}


% 表格與圖片
\usepackage{longtable, booktabs, threeparttable, graphicx, subfigure, epstopdf}

% 其他設置
\usepackage{xcolor, comment, indentfirst, enumitem, titlesec}
\usepackage{algorithm, algorithmic}
\usepackage{url, cite}
\usepackage{xeCJK}
\usepackage{CJKnumb}

% Times New Roman 字體
\usepackage[T1]{fontenc}
% \usepackage[utf8]{inputenc}
\usepackage{mathptmx}

% 設置字體
% \setCJKmainfont{AR PL UKai TW}
% \setCJKmainfont[AutoFakeBold=4,AutoFakeSlant=.4]{AR PL UKai TW}
\setCJKmainfont{標楷體}



% 設置頁眉頁腳
% \pagestyle{fancy}
% \fancyhf{}
% \rhead{\thepage / \pageref{LastPage}}
% \lhead{\textbf{論文標題}}
% \cfoot{}

% 重新定義章節標題格式
\titleformat{\section}{\bf\Large}{\CJKnumber{\arabic{section}}、}{1em}{}

\renewcommand{\figurename}{圖}

\begin{document}
% \fontspec{?} 
% % 封面
% \begin{center}
%     {\Large \textbf{論文標題}}\\[1.5em]
%     {\large 作者姓名}\\[1em]
%     {\large 日期}
% \end{center}

\vspace{2em}

% 摘要
% \begin{abstract}
%     本文探討了衛星與地面站之間的糾纏分發問題,並建立了相應的數學模型。我們分析了傳輸距離、保真度與信道衰減的影響,並提出了一種最佳化調度方法。數值結果顯示,所提出的方法能有效提高系統性能。
% \end{abstract}

\vspace{2em}

% 主要符號與意義
\section{主要符號與意義列表}
\begin{longtable}{p{0.15\textwidth} p{0.8\textwidth}}
\caption{論文中主要符號與其物理/數學意義}\\
\toprule
\textbf{符號} & \textbf{意義} \\
\midrule
\endfirsthead
\toprule
\textbf{符號} & \textbf{意義} \\
\midrule
\endhead
\midrule
\multicolumn{2}{r}{(下頁繼續)}
\endfoot
\bottomrule
\endlastfoot

\textbf{\(S\)} & 衛星集合\\
\textbf{\(G\)} & 地面站集合\\
\textbf{\(F\)} & 需要共享糾纏的地面站對集合\\
\textbf{\(N\)} & 衛星總數(\(\lvert S\rvert\))\\
\textbf{\(M\)} & 地面站對總數(\(\lvert F\rvert\))\\
\textbf{\(\theta_e\)} & 最低仰角\\
\textbf{\(\kappa\)} & 時槽總數\\
\textbf{\(\Delta\)} & 每時槽長度(秒)\\
\textbf{\(t\)} & 時槽索引\\
\textbf{\(e_{ig}(t)\)} & 衛星 \(i\) 與地面站 \(g\) 的仰角\\
\textbf{\(s_{ig}(t)\)} & 衛星 \(i\) 到地面站 \(g\) 的距離\\
\end{longtable}

% % 方法部分
% \section{方法}
% 本節介紹衛星-地面站糾纏分發的數學模型與最佳化方法。

% % 實驗結果
% \section{實驗結果}
% 本節展示我們的數值模擬結果,並分析其對系統性能的影響。

% % 結論
% \section{結論}
% 本文提出了一種基於最佳化的衛星糾纏分發方法,並驗證了其有效性。

% 參考文獻
% \begin{thebibliography}{99}
%     \bibitem{ref1} J. Smith, ``Quantum Networks,'' \textit{Journal of Quantum Science}, vol. 1, no. 1, pp. 1-10, 2024.
%     \bibitem{ref2} A. Johnson, ``Entanglement Distribution in Space,'' \textit{IEEE Transactions on Quantum Engineering}, vol. 5, pp. 50-60, 2023.
% \end{thebibliography}

\end{document}
